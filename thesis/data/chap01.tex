\chapter{绪论}

\section{研究背景和动机}

人类的视觉感知系统存在有限的感知域。对于超出感知域的变化,裸眼无法感
知\upcite{deering1998limits}。然而,这类信号却可能带有重要的信息。

例如,血液循环使得人体的皮肤发生细微的周期性变化,这个裸眼无法感知的变化却和人的
心率非常吻合\upcite{verkruysse2008remote,poh2010non}。再比如,乐器在演奏过程中会
产生微弱的形变,而这个形变的频率却和乐器的音高保持一致。

影像动作放大技术是近年来被提出的一种用于改变影像中感兴趣物体的变化幅度的技术,这
类技术有如动作信息的“显微镜”,可以将这些微弱的信号放大到肉眼可以感知的幅度,或
者可以对已有的变化进行调整,从而挖掘出有价值的信息,如心率的估计,对动画场景的夸
张处理等。影像动作放大技术在医学、军事、刑侦、遥感、动画影像制作等领域都有着广阔
的应用价值。

对影像动作放大技术的研究可以追溯到有关捕捉、操纵和重放动作数据的研究。(\textcolor{red}{todo: 这里将列举和分析文献\cite{wang1994representing,Unuma1995,Gleicher1998,Lee2002,Brand2000,Pullen2002,Li2002,Jojic2001,brostow1999motion}的内容。})

有别于以上所提及的技术,影像动作放大技术直接通过处理视频图像数据来将微弱的动作信号放
大,而非通过修改标记点的位置信息来达到放大动作的目的,因而具有更大的难度。

2005年,Liu等人最早提出了一种影像动作放大技术\upcite{liu2005motion},该方法首先对
视频图像进行对齐,将经过对齐的图像的特征点进行聚类,并跟踪这些点随时间的运动轨迹,
从而得到不同的动作层,最后将用户选定的动作层的运动幅度加大。Liu的方法属于典型的
拉格朗日视角的方法,即从跟踪粒子的运动轨迹的角度着手分析。

不同于拉格朗日视角的方法,Wu等人在2012年提出了一种称为欧拉影像放大技术(Eulerian
Video Magnification)的方法\upcite{wu2012eulerian},该方法站在欧拉的视角,并不显
式地跟踪和估计粒子的运动,通过分析整个场景图像的像素点的值随时间的变化,从中
分离出感兴趣的频段并进行增强,从而达到放大动作信息的目的。

线性的欧拉影像动作放大方法会在放大动作的同时放大噪声,因
此,Wadhwa等人在2013年对欧拉影像放大技术进行了改进,提出了基于相位的影像动作处
理技术\upcite{Wadhwa2013PhaseBased}。基于相位的欧拉影像放大技术在放大动作的
同时不会放大噪声,而是平移了噪声,因而可以达到更好的放大效果。

影像动作处理技术自提出起就被广泛应用在影像处理、动画制作、医学等领域,且在军事、
刑侦、遥感等领域都有着可以预见的应用价值。2006年,Wang等人基于Liu的方法设计了
一个卡通动画滤镜,可以调整动画场景中目标物体的动作幅
度\upcite{wang2006cartoon}。2012年,Wu等人基于欧拉影像动作放大技术设计了一个免接
触式心率检测系统\upcite{wu2012phd},直接利用摄像头采集到的数据来估算实验者的心率。
在具备良好的光照条件,以及实验者没有大幅移动身体的情况下,该系统可以获得临床级别
的准确度。2013年,Didyk等人在基于相位的欧拉影像动作处理技术的基础上提出了一个将
立体影片中每一帧中的两个视角的图像扩展到多个视角的图像的算法\upcite{Didyk2013},
该算法可以自动地将每一帧两个视角的普通3D电影扩展为每一帧8个视角的裸眼3D电影。

\section{本文工作及创新点}
\label{sec:creation}

本文提出了一种结合了拉格朗日视角和欧拉视角的优点的影像动作放大方法,该方法基于欧
拉影像动作放大方法,从拉格朗日视角的方法得到启发,将变化信号的处理范围约束在由用
户选择的感兴趣区域上,可以有效的避免“鬼影”问题,改进放大结果。

本文的主要创新点包括以下三点:

\begin{compactenum}
\item %不对整个场景进行放大,而是将放大区域限制在由用户选定的感兴趣区域上。\\
  针对欧拉影像动作放大技术在场景中存在较大幅度动作时容易出现“鬼影”效应,及易受
  非感兴趣区域的变化信号干扰的问题,本文从基于拉格朗日视角的方法得到启发,通过
  使用Mean-shift算法进行目标跟踪,将放大区域限制在由用户选定的感兴趣区域上。
  实验结果表明,该方法应用在已存在大幅度变化的场景时可以有效的避免“鬼影”问题。
  此外,将放大的区域限制在感兴趣的区域,可以减少场景中的其他部分对该区域的干扰,
  有利于对放大结果的后续分析。
\item %不直接对放大后的变化与源图像进行线性叠加,而是将经过放大的动作与感兴趣区域
  %的前景部分进行多分辨率混合。\\
  欧拉影像放大方法会将视频场景中的背景部分连同前景部分一块被放大,因此区域放大
  的结果会出现明显的边界。本文通过使用GrabCut算法,从目标区域中进一步提取出前景
  的掩码,再根据这个掩码,将放大结果与原图进行金字塔混合,可以有效地抑制背景部分
  的放大,从而去除边界处的失真问题。
\item 对动作变化的放大会导致出现颜色噪点,为了抑制颜色噪点,以往的做法是先将图像
  由RGB颜色空间转换到YIQ颜色空间,然后对I、Q分量进行衰减。本文改为将图像
  由RGB颜色空间转换到更符合人眼感知特性的CIEL*a*b*颜色空间,在放大后对a*、
  b*分量进行衰减,可以达到较好的去噪效果。
\item %不分别处理视频彩色序列帧的三个通道,而是借助四元数同时处理三个通道。\\
  在进行带通滤波时,不分别对视频的彩色序列帧的三个颜色通道进行单独的处理,而是利
  用四元数这个数学工具,将L*、a*、b*三个通道用一个四元数矩阵来表示,一次性对整
  个矩阵进行滤波处理,具备更好的整体性。
\end{compactenum}

% 本文的内容结构如下:

% 第一章~绪论。介绍研究背景和动机,总结国内外的研究现状,并阐述了本文的工作及创新
% 点。

% 第二章~影像动作放大技术概述。介绍经典的影像动作放大技术,包括拉格朗日视角的方法,
% 以及欧拉视角的方法。并分析每种技术的优劣势。

% 第三章~前景约束的欧拉影像动作放大技术。介绍本文提出的前景约束的欧拉影响动作放大
% 方法的算法框架和流程。

% 第四章~实验结果。介绍具体的实现细节,并与线性的欧拉影像动作放大技术进行实验结果
% 对比。

% 第五章~总结与展望。

%%% Local Variables: 
%%% mode: latex
%%% TeX-master: "../thesis"
%%% End: 
