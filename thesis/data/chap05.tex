\chapter{总结与展望}
\label{chap:conclusion}

影像动作放大技术是近年来被提出的一种用于改变影像中感兴趣物体的变化幅度的技术,该
技术自提出起就被广泛应用在影像处理、动画制作、医学等领域,且在军事、刑侦、遥感等
领域都有着可以预见的应用价值。

本文总结了现有的两种视角\pozhehao 拉格朗日视角和欧拉视角下的影像动作放大方法各自
的特点和不足。针对拉格朗日视角的影像动作放大算法的复杂度高、易受遮挡影响,而欧拉
视角下的影像动作放大算法在放大存在大幅度动作的物体的变化时容易产生“鬼影”的问题,
本文提出一种前景约束的欧拉影像动作放大方法,该方法结合了两种视角的方法的特点,通
过使用Mean-shift与Kalman滤波相结合的目标跟踪技术,从原视频序列中提取出一个由目标
区域组成的新序列,进而对这个序列进行局部的欧拉影像放大。同时,使用GrabCut算法从原
视频图像帧中提取出目标区域的前景掩码。再根据这个掩码,将放大结果与原图进行金字塔
混合,可以有效地抑制背景部分的放大,从而去除边界处的失真问题。

在使用欧拉影像动作放大算法进行局部动作放大时,为了抑制颜色噪点,本文通过先将图像
由RGB颜色空间转换到更符合人眼感知特性的CIEL*a*b*颜色空间,在放大后对a*、b*分量进
行衰减,可以达到较好的去噪效果。此外,在进行带通滤波时,不分别对视频的彩色序列帧
的三个颜色通道进行单独的处理,而是利用四元数这个数学工具,将L*、a*、b*三个分量放
在四元数的矢量部分,构成一个纯四元数,再进行基于四元数的时频处理,具备更好的整体
性。

为了验证本文方法的有效性,本文在实验部分设计了四个实验。前三个实验分别对三个存在
不同动作幅度的案例视频进行动作放大。通过与线性的欧拉影像动作放大方法进行结果对比,
可以看出使用本文的算法可以更好的放大感兴趣区域的变化,且不会产生明显的“鬼影”问
题。

在第四个实验中,本文结合了全局的欧拉影像动作放大方法和本文的方法对案例视频进行了
两步放大,通过实验结果证明了两种方法的结合使用既可以放大场景中存在大幅度动作的物
体中的细微动作,也可以放大场景中的其他区域。

但是,目前所做的工作还有需要完善的地方。

\begin{compactenum}
\item 前景分割算法的优化。本文的算法主要耗时在于前景分割模块。而基于GrabCut的分
  割算法是逐帧进行的,没有考虑到帧与帧之间的相关性。在未来的工作中,可以利用帧与
  帧之间的相关性减少算法的迭代次数。
\item 跟踪算法的增强。Mean-shift算法在跟踪过程中由于窗口宽度大小保持不变,当目标
  尺度有所变化时,跟踪就会失败。在未来的工作中,可以对现有的跟踪算法进行增强,如
  改用带宽动态变化的CamShift算法。
\item  引入视频校准算法。使用Mean-shift进行跟踪得到的目标区域不可避免会存在抖动,
  影响放大效果。在未来的工作中,可以在完成目标跟踪后,对目标区域组成的序列进行视
  频校准,以消除因跟踪窗口的抖动造成的动作。
\end{compactenum}

%%% Local Variables: 
%%% mode: latex
%%% TeX-master: "../thesis"
%%% End: 
