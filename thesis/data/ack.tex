\begin{ack}
  % 在读研前,一位老师向我形容读研的感受:“本科的时候,我不知道知识的门在哪里。读
  % 完研,我看到了那扇门。”怀着对知识的向往与崇敬,我也踏上了寻门之路。三年的读研
  % 时光,给了我一个静心学习的机会。临近毕业之际,重新品读这句话,幡然醒悟:知识的
  % 大门,就在灯火阑珊处。只有戒骄戒躁,踏实研究,才能发现这扇神奇的门。

  感谢我的恩师李兴民教授。李教授为人谦逊热忱,治学严谨,不但有深厚的学术造诣,对
  学生的关怀也无微不至。在他的悉心指导下,我不仅在学术研究上有所长进,更在为人处
  世方面受益匪浅。在毕业论文写作期间,李教授给我提出了很多宝贵意见,帮助我顺利完
  成论文工作。今后,我会谨记李教授对我的指导,学习他认真专注的科研精神,学习他乐
  观豁达的心境,铭记他“知足知不足,有为有不为”的教诲,不负师恩。

  感谢计算机学院一路走来指导过我的其他老师,尤其是鲍苏苏教授、单志龙教授、王立斌
  副教授、陈寅副教授以及张奇支副教授,他们是我求知路上的明灯,用渊博的学识授人以
  渔。在他们的身上,我不仅学到丰富的专业技能,也深深感受到分享和交流所带来的
  快乐。感谢我的师母王敬老师以及辅导员谢子娟老师,她们在生活和工作中给予我细心爱
  护和帮助。

  感谢我的父母,他们用勤劳的双手为我们经营起一个温馨的家庭。生活纵有风浪,但我们
  却更加相亲相爱。感谢我的哥哥和两个妹妹,谢谢他们对我的鼓励和包容。他们一直都是
  我前行的坚实后盾。
  
  感谢深圳先进技术研究院的老师和同学,感谢陈宝权教授,Daniel Cohen-Or教
  授,Oliver Deaussen教授和Andrei Sharf博士,在实习的期间,他们让我开阔了眼界,见
  识到科研的魅力,意识到团队的力量。

  感谢我的同门,谢谢他们在求知路上与我一路同行,谢谢他们对我的信任与支持。感谢我
  的舍友彭俊、朱沐青和林泽殷,谢谢他们一同为B618宿舍创造了一个温馨舒适的环境。感
  谢赵仕豪、徐泽坤、李建铭师弟,和他们的科研项目合作得很愉快。还要感谢雷楚楚师妹,
  谢谢她为我提供了实验视频,并帮我校对论文。

  我的论文也离不开开源社区所提供的帮助。感谢GNU、Linux、Github、OpenCV。今后我也
  将尽我所能,争取在开源社区多做贡献。

  最后,感谢参与我论文评审和答辩的各位专家,谢谢你们的认真聆听和宝贵意见!
  
  % 学海无涯,三年的研究生时光让我的专业水平更上一层,但这只是一个起点。我的导师李
  % 兴民教授曾在他的课上写了一行“路漫漫其修远兮”的诗句送给我们。未来的日子,我会
  % 活到老,学到老。争取做一个对社会有用的人。
\end{ack}
%%% Local Variables:
%%% mode: latex
%%% TeX-master: "../thesis"
%%% End:

