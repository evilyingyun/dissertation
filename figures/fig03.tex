\documentclass{article}

\usepackage[english]{babel}
\usepackage[latin1]{inputenc}
\usepackage{times}
\usepackage[T1]{fontenc}
\usepackage{tabularx,multirow,multicol,keystroke,subfigure}

% CJK packages
\usepackage{xeCJK}
\setCJKmainfont{SimHei}

\usepackage{tikz}
\usepackage[xelatex,active,tightpage]{preview}

\setlength\PreviewBorder{0mm}
\PreviewEnvironment{tabular}

\usetikzlibrary{arrows,shapes.misc,positioning,calc,chains,scopes}

\tikzstyle{neuron}=[circle,fill=black!25,minimum size=12,inner sep=4]
\tikzstyle{input neuron}=[neuron, fill=green!25, draw=green!70]
\tikzstyle{output neuron}=[neuron, fill=blue!25, draw=blue!70]
\tikzstyle{hidden neuron}=[neuron, fill=red!25, draw=red!70]
\tikzstyle{small neuron}=[circle, minimum size=6, inner sep=2, fill=blue!25, draw=blue!70]
\tikzstyle{A neuron}=[circle, minimum size=6, inner sep=0, fill=green!25, draw=green!70]
\tikzstyle{B neuron}=[circle, minimum size=6, inner sep=0, fill=red!25, draw=red!70]
\tikzstyle{ouput box}=[rectangle, fill=blue!25, draw=blue!70, minimum size=12, inner sep=4]
\tikzstyle{hidden box}=[rectangle, fill=red!25, draw=red!70, minimum size=12, inner sep=4]
\tikzstyle{thickarrow}=[thick,>=stealth]
\tikzstyle{connect}=[thin]

\begin{document}


  \begin{tabular}[t]{c|l|c|c}
    \hline
    结构 &  \hspace{5em} 决策区域类型  & 区域形状 & 异或问题 \\
    \hline
    \begin{tikzpicture}[node distance=0.5 and 0.5]
      \tiny
      \node (A)[small neuron]{};
      \node (B)[small neuron,right=of A]{};

      \node (X) at ($ (A)!.5!(B) $) {};

      \node (C)[small neuron,label=above:无隐层] at ($ (X) ! {sin(60)*2} ! 90:(B) $) {}
      edge[<-](A)
      edge[<-](B);
    \end{tikzpicture}
    &
    由一超平面分成两个
    &
    \begin{tikzpicture}
      \draw (0,0)--(1,0)--(1,1)--(0,1)--cycle;
      \fill [black] (0,0)--(1,0)--(1,0.2)--(0.2,0.8)--(0,0.8)--cycle;
    \end{tikzpicture}
    &
    \begin{tikzpicture}
      \tiny
      \begin{scope}
        \shadedraw[clip,top color=gray,bottom color=blue!10]
        (0,0.3)--(0.7,1)--(0,1)--cycle;
        \foreach \x in {0.1,0.2,...,1.0}
        \draw[rotate=45] (\x,2) -- (\x,-2);
      \end{scope}
      \node [A neuron] at (0.25,0.75){A};
      \node [A neuron] at (0.75,0.25){A};

      \node [B neuron] at (0.75,0.75){B};
      \node [B neuron] at (0.25,0.25){B};

      \draw (0,0)--(1,0)--(1,1)--(0,1)--cycle;
    \end{tikzpicture}
    \\
    \hline
    \begin{tikzpicture}[node distance=0.2 and 0.4]
      \tiny
      \node (A)[small neuron]{};
      \node (B)[small neuron,right=of A]{};

      \node (E)[small neuron,below=of A]{}
      edge[->](A)
      edge[->](B);
      \node (F)[small neuron,right=of E]{}
      edge[->](A)
      edge[->](B);

      \node (X) at ($ (A)!.5!(B) $) {};

      \node (C)[small neuron,label=above:单隐层,yshift=10] at (X) {}
      edge[<-](A)
      edge[<-](B);
    \end{tikzpicture}
    & 开凸区域或闭凸区域
    &
    \begin{tikzpicture}
      \draw (0,0)--(1,0)--(1,1)--(0,1)--cycle;
      \fill (0,0)--(1,0)--(0.3,0.4)--(0.2,0.6)--(0.5,1)--(0,1)--cycle;
    \end{tikzpicture}
    &
    \begin{tikzpicture}
      \tiny
      \begin{scope}
        \shadedraw[clip,top color=gray,bottom color=blue!10]
        (0,0.7)--(0.7,0)--(1,0)--(1,0.3)--(0.3,1)--(0,1)--cycle;
        \foreach \x in {-1.0,-0.9,...,1.0}
        \draw[rotate=135] (\x,2) -- (\x,-2);
      \end{scope}
      \node [A neuron] at (0.25,0.75){A};
      \node [A neuron] at (0.75,0.25){A};

      \node [B neuron] at (0.75,0.75){B};
      \node [B neuron] at (0.25,0.25){B};

      \draw (0,0)--(1,0)--(1,1)--(0,1)--cycle;
    \end{tikzpicture}
    \\
    \hline
    \begin{tikzpicture}[node distance=0.2 and 0.4]
      \tiny
      \node (i1)[small neuron]{};
      \node (i2)[small neuron,right=of i1]{};

      \node (X) at ($ (i1)!.5!(i2) $) {};

      \node (h1)[small neuron,above left=of i1]{}
      edge[<-](i1)
      edge[<-](i2);

      \node (h2)[small neuron,above=of X]{}
      edge[<-](i1)
      edge[<-](i2);

      \node (h3)[small neuron,above right=of i2]{}
      edge[<-](i1)
      edge[<-](i2);

      \node (H1)[small neuron,above=of i1,yshift=10]{}
      edge[<-](h1)
      edge[<-](h2)
      edge[<-](h3);

      \node (H2)[small neuron,above=of i2,yshift=10]{}
      edge[<-](h1)
      edge[<-](h2)
      edge[<-](h3);

      \node (O)[small neuron, above=of h2,yshift=10,label=above:双隐层]{}
      edge[<-](H1)
      edge[<-](H2);


    \end{tikzpicture}
    &
    任意形状(其复杂度由单元数目确定)
    &
    \begin{tikzpicture}
      \tiny
      \fill (0,0)--(1,0)--(1,1)--(0,1)--cycle;
      \fill [white] (0.2,0.6)--(0.23,0.57)--(0.4,0.7)
      --(0.27,0.8)--(0.21,0.75)--(0.16,0.73)--cycle;
      \fill [white] (0.6,0.18)--(0.8,0.15)--(0.9,0.2)--(0.92,0.43)
      --(0.8,0.65)--(0.65,0.6)--(0.7,0.43)--(0.5,0.35)--cycle;
      \fill (0.65,0.26)--(0.78,0.22)--(0.88,0.35)--(0.8,0.45)
      --(0.7,0.40)--cycle;
    \end{tikzpicture}
    &
    \begin{tikzpicture}
      \tiny
      \begin{scope}
        \shadedraw[clip,top color=gray,bottom color=blue!10]
        (0.05,0.85)--(0.02,0.5)--(0.45,0.6)--(0.35,0.92)--cycle;
        \foreach \x in {-1.0,-0.9,...,1.0}
        \draw[rotate=135] (\x,2) -- (\x,-2);
      \end{scope}
      \begin{scope}
        \shadedraw[clip,top color=gray,bottom color=blue!10]
        (0.55,0)--(1,0)--(1,0.4)--(0.75,0.5)--(0.5,0.45)--cycle;
        \foreach \x in {-1.0,-0.9,...,1.0}
        \draw[rotate=135] (\x,2) -- (\x,-2);
      \end{scope}

      \node [A neuron] at (0.25,0.75){A};
      \node [A neuron] at (0.75,0.25){A};

      \node [B neuron] at (0.75,0.75){B};
      \node [B neuron] at (0.25,0.25){B};

      \draw (0,0)--(1,0)--(1,1)--(0,1)--cycle;
    \end{tikzpicture}
    \\
    \hline
  \end{tabular}
\end{document}

%%% Local Variables: 
%%% mode: latex
%%% TeX-master: t
%%% End: 
